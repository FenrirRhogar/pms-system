\documentclass[12pt,a4paper]{article}

% Packages for LuaLaTeX/XeLaTeX
\usepackage{fontspec}
\usepackage{polyglossia}
\setmainlanguage{greek}
\setotherlanguage{english}

% Font setup - Make sure a font with Greek support is available
% Linux Libertine O is a common choice for LuaLaTeX
\setmainfont{Linux Libertine O}
\setsansfont{Linux Biolinum O}

\usepackage{graphicx}
\usepackage{hyperref}
\usepackage{listings}
\usepackage{xcolor}
\usepackage{geometry}
\geometry{margin=2.5cm}

% Code highlighting setup
\lstset{
    basicstyle=\ttfamily\small,
    breaklines=true,
    frame=single,
    backgroundcolor=\color{gray!10},
    keywordstyle=\color{blue},
    commentstyle=\color{green!50!black},
    stringstyle=\color{red}
}

\title{\textbf{Αναφορά Εργασίας: Project Management System}}
\author{
    [Ονοματεπώνυμο Φοιτητή 1] (ΑΜ: ...) \\
    [Ονοματεπώνυμο Φοιτητή 2] (ΑΜ: ...)
}
\date{\today}

\begin{document}

\maketitle

\tableofcontents
\newpage

\section{Περιγραφή Εφαρμογής}
Η παρούσα εργασία αφορά την ανάπτυξη ενός ολοκληρωμένου Συστήματος Διαχείρισης Έργων (Project Management System - PMS). Η εφαρμογή είναι σχεδιασμένη με αρχιτεκτονική μικροϋπηρεσιών (Microservices) και επιτρέπει τη συνεργασία ομάδων, τη διαχείριση χρηστών και την ανάθεση/παρακολούθηση εργασιών.

Οι βασικές λειτουργικές ενότητες περιλαμβάνουν:
\begin{itemize}
    \item \textbf{Διαχείριση Χρηστών:} Εγγραφή, σύνδεση, αυθεντικοποίηση (JWT) και ρόλοι χρηστών (Admin, Team Leader, Member).
    \item \textbf{Διαχείριση Ομάδων:} Δημιουργία ομάδων από διαχειριστές και ανάθεση μελών.
    \item \textbf{Διαχείριση Εργασιών:} Δημιουργία, ενημέρωση, ανάθεση εργασιών και σχολιασμός.
    \item \textbf{Admin Panel:} Κεντρική διαχείριση για ενεργοποίηση χρηστών και εποπτεία του συστήματος.
\end{itemize}

\section{Αρχιτεκτονική και Τεχνολογίες}
Η εφαρμογή ακολουθεί το μοντέλο των Microservices, όπου κάθε υπηρεσία εκτελείται σε απομονωμένο container.

\subsection{Microservices}
\begin{enumerate}
    \item \textbf{User Service (Port 8080):} Υπεύθυνο για την εγγραφή, το login και τη διαχείριση προφίλ χρηστών. Χρησιμοποιεί JWT για την ασφάλεια.
    \item \textbf{Team Service (Port 8081):} Διαχειρίζεται τις ομάδες και τη συμμετοχή μελών σε αυτές.
    \item \textbf{Task Service (Port 8082):} Διαχειρίζεται τα tasks και τα σχόλια (comments).
\end{enumerate}

\subsection{API Gateway & Frontend}
\begin{itemize}
    \item \textbf{API Gateway (Nginx):} Δρομολογεί τα αιτήματα από το frontend προς τα κατάλληλα microservices (Reverse Proxy).
    \item \textbf{Frontend (React.js):} Single Page Application (SPA) που προσφέρει το περιβάλλον αλληλεπίδρασης με τον χρήστη.
\end{itemize}

\subsection{Βάση Δεδομένων}
Χρησιμοποιήθηκε η πλατφόρμα \textbf{Supabase} (PostgreSQL) ως κεντρική βάση δεδομένων για την αποθήκευση χρηστών, ομάδων και εργασιών. Επιλέχθηκε για την ταχύτητα ανάπτυξης και την αξιοπιστία της, αντικαθιστώντας τον συνδυασμό MySQL/MongoDB.

\section{Υλοποίηση}

\subsection{Τι υλοποιήθηκε}
\begin{itemize}
    \item Πλήρης κύκλος ζωής χρήστη (Εγγραφή -> Ενεργοποίηση από Admin -> Login).
    \item Role-Based Access Control (RBAC) για Admin, Team Leader, Member.
    \item Λειτουργικότητα CRUD για Ομάδες και Tasks.
    \item Σύστημα σχολίων (Comments) εντός των εργασιών.
    \item Δυναμικό UI με React Hooks και Context.
    \item Containerization όλων των υπηρεσιών με Docker και Docker Compose.
    \item Inter-service communication (σύγχρονη επικοινωνία μέσω HTTP REST calls).
\end{itemize}

\subsection{Τι δεν υλοποιήθηκε / Παραλείψεις}
\begin{itemize}
    \item \textbf{Διαχωρισμός Βάσεων:} Αντί για χρήση ξεχωριστών βάσεων MySQL και MongoDB, χρησιμοποιήθηκε ενιαία PostgreSQL (μέσω Supabase) για απλοποίηση της διαχείρισης δεδομένων στο Cloud.
    \item \textbf{Ειδοποιήσεις (Notifications):} Δεν υλοποιήθηκε σύστημα ειδοποιήσεων (Real-time ή σύγχρονο list) λόγω χρονικών περιορισμών.
    \item \textbf{Ασύγχρονη Επικοινωνία:} Δεν χρησιμοποιήθηκε Kafka/RabbitMQ (Bonus feature).
\end{itemize}

\subsection{Ιδιαίτερα Στοιχεία Υλοποίησης}
\begin{itemize}
    \item \textbf{Ασφάλεια Admin:} Υλοποιήθηκε μηχανισμός ασφαλείας που αποτρέπει τη διαγραφή ή την απενεργοποίηση διαχειριστών (Admins) από άλλους διαχειριστές, για την αποφυγή "κλειδώματος" του συστήματος.
    \item \textbf{Deep Linking:} Το Frontend υποστηρίζει απευθείας συνδέσμους (deep links) για ομάδες και εργασίες, διευκολύνοντας την πλοήγηση.
\end{itemize}

\section{Migration & Deployment στο Google Cloud (GCP)}
Η διαδικασία μεταφοράς της εφαρμογής στο Cloud περιλάμβανε τα εξής βήματα:

\begin{enumerate}
    \item \textbf{Δημιουργία VM Instance:} Δημιουργήθηκε instance (e2-medium) με λειτουργικό σύστημα Ubuntu 20.04 LTS στο Google Compute Engine.
    \item \textbf{Προετοιμασία Περιβάλλοντος:}
    \begin{lstlisting}[language=bash]
sudo apt-get update
sudo apt-get install docker.io docker-compose
    \end{lstlisting}
    \item \textbf{Μεταφορά Κώδικα:} Ο κώδικας μεταφέρθηκε στο VM μέσω `git clone` από το αποθετήριο.
    \item \textbf{Ρύθμιση Περιβάλλοντος:} Δημιουργία αρχείου `.env` με τα credentials της Supabase και τα JWT secrets.
    \item \textbf{Εκκίνηση Υπηρεσιών:}
    \begin{lstlisting}[language=bash]
sudo docker-compose up --build -d
    \end{lstlisting}
    \item \textbf{Ρύθμιση Firewall:} Άνοιγμα των θυρών στο VPC network του GCP:
    \begin{itemize}
        \item TCP 3000 (Frontend)
        \item TCP 80 (API Gateway)
    \end{itemize}
\end{enumerate}

\section{Οδηγίες Πρόσβασης}

\textbf{Τοπική Εκτέλεση:}
\begin{itemize}
    \item \textbf{Frontend:} \url{http://localhost:3000}
    \item \textbf{API Gateway:} \url{http://localhost:80}
    \item \textbf{Swagger Docs (User Service):} \url{http://localhost:8080/docs}
\end{itemize}

\textbf{Cloud Deployment (Ενδεικτικά):}
\begin{itemize}
    \item \textbf{URL:} \texttt{http://<GCP-EXTERNAL-IP>:3000}
\end{itemize}

\end{document}
